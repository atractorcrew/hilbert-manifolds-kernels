\documentclass[12pt]{article}
\usepackage[utf8]{inputenc}
\usepackage{amsmath,amssymb,amsthm}
\usepackage{array}
\renewcommand{\arraystretch}{1.35}
\setlength{\tabcolsep}{8pt}

% Columna cómoda para texto en p{} con cortes de línea seguros
\newcolumntype{L}[1]{>{\raggedright\arraybackslash}p{#1}}

\title{Caja de Herramientas: Espacios Métricos e Inequidades}
\author{}
\date{}

\begin{document}
\maketitle

\section*{Resumen}

\begin{tabular}{|L{4.3cm}|L{11.5cm}|}
\hline
\textbf{Espacio métrico} &
Un par \((X,d)\) donde \(d:X\times X\to\mathbb{R}_{\ge 0}\) cumple:
(i) \(d(x,y)=0\iff x=y\),
(ii) \(d(x,y)=d(y,x)\),
(iii) \textbf{desigualdad triangular} \(d(x,z)\le d(x,y)+d(y,z)\).
\\
\hline

\textbf{Bolas y topología} &
\textit{Bola abierta}: \(B(x,r)=\{y:\ d(x,y)<r\}\).
\; \textit{Bola cerrada}: \(\overline{B}(x,r)=\{y:\ d(x,y)\le r\}\).
\; Los abiertos son uniones de bolas abiertas.
\\
\hline

\textbf{Cierre / interior / frontera} &
\(x\in\overline{A}\iff \forall r>0,\ B(x,r)\cap A\neq\varnothing\).
\; \(\operatorname{int}(A)=\{x:\ \exists r>0,\ B(x,r)\subset A\}\).
\; \(\partial A=\overline{A}\setminus \operatorname{int}(A)\).
\\
\hline

\textbf{Convergencia} &
\(x_n\to x \iff \forall\varepsilon>0\ \exists N\ \forall n\ge N:\ d(x_n,x)<\varepsilon\).
\; \textbf{Caracterización de cierre}: \(\overline{A}=\{x:\ \exists (a_n)\subset A,\ a_n\to x\}\).
\\
\hline

\textbf{Continuidad} &
\(f:(X,d_X)\to(Y,d_Y)\) es continua en \(x_0\) si
\(\forall\varepsilon>0\ \exists\delta>0:\ d_X(x,x_0)<\delta\Rightarrow d_Y(f(x),f(x_0))<\varepsilon\).
\; Equivalente: \(x_n\to x\Rightarrow f(x_n)\to f(x)\).
\\
\hline

\textbf{Ineq. triangular (básica)} &
\(d(x,z)\le d(x,y)+d(y,z)\) (toda métrica).
\; \textbf{Iterada}: \(d(x_0,x_n)\le \sum_{k=1}^n d(x_{k-1},x_k)\).
\\
\hline

\textbf{Ineq. triangular inversa} &
\(\lvert d(x,z)-d(y,z)\rvert\le d(x,y)\) (toda métrica).
\; Consecuencia: \(\lvert d(x,A)-d(y,A)\rvert\le d(x,y)\).
\\
\hline

\textbf{Distancia a un conjunto} &
\(d(x,A):=\inf\{d(x,a):a\in A\}\) (convención \(d(x,\varnothing)=+\infty\)).
\par
Propiedades clave:
\begin{itemize}
\item \(0\le d(x,A)\le d(x,a)\) para todo \(a\in A\).
\item \(d(x,A)=0\iff x\in\overline{A}\).
\item \textbf{1-Lipschitz}: \(|d(x,A)-d(y,A)|\le d(x,y)\).
\item Si \(A\) es cerrado y \(d(x,A)>0\), entonces \(B(x,d(x,A))\cap A=\varnothing\).
\end{itemize}
\\
\hline

\textbf{Diámetro} &
\(\operatorname{diam}(A):=\sup\{d(x,y):x,y\in A\}\).
\; Si \(A\subset B\), entonces \(\operatorname{diam}(A)\le \operatorname{diam}(B)\).
\; Si \(f\) es \(L\)-Lipschitz, \(\operatorname{diam}(f(A))\le L\,\operatorname{diam}(A)\).
\\
\hline

\textbf{Aplicaciones Lipschitz} &
\(f\) es \(L\)-Lipschitz si \(d_Y(f(x),f(y))\le L\,d_X(x,y)\).
\; Si \(L<1\) (contracción) \(\Rightarrow\) punto fijo de Banach (en completos).
\\
\hline

\textbf{Secuencias de Cauchy} &
\((x_n)\) es Cauchy si \(\forall\varepsilon>0\ \exists N:\ m,n\ge N\Rightarrow d(x_m,x_n)<\varepsilon\).
\; Toda convergente es Cauchy; en espacios \textbf{completos}, toda Cauchy converge.
\\
\hline

\textbf{Compacidad (métricos)} &
En métricos: \textit{compacto} \(\Leftrightarrow\) \textit{secuencialmente compacto} \(\Leftrightarrow\) \textit{totalmente acotado + completo}.
\; Totalmente acotado: para todo \(\varepsilon>0\) existe recubrimiento finito por bolas de radio \(\varepsilon\).
\\
\hline

\textbf{Métricas producto} &
Si \(d_i\) son métricas en \(X_i\), en \(X_1\times X_2\):
\(\ d_{\max}((x_1,x_2),(y_1,y_2))=\max\{d_1(x_1,y_1),d_2(x_2,y_2)\}\),
\(\ d_{1}((x_1,x_2),(y_1,y_2))= d_1(x_1,y_1)+d_2(x_2,y_2)\).
\; Ambas inducen la topología producto.
\\
\hline

\textbf{Métricas inducidas por norma} &
Si \(\|\cdot\|\) es norma, \(d(x,y)=\|x-y\|\) es métrica.
\; \textbf{Minkowski}: \(\|x+y\|\le \|x\|+\|y\|\).
\; \textbf{Triangular inversa}: \(\lvert\|x\|-\|y\|\rvert\le \|x-y\|\).
\; En espacios con producto interno: \(\lvert\langle x,y\rangle\rvert\le \|x\|\,\|y\|\) (Cauchy–Schwarz).
\\
\hline

\textbf{Ineq. con bolas} &
Si \(y\in B(x,r)\), entonces \(B(y,\rho)\subset B(x,r+\rho)\).
\; Si \(B(x,r)\cap B(y,s)=\varnothing\), entonces \(d(x,y)\ge r+s\).
\; Si \(z\in B(x,r)\cap B(y,s)\), entonces \(d(x,y)\le r+s\).
\\
\hline

\textbf{Separación por cerrados} &
Si \(A\) es cerrado y \(x\notin A\), entonces \(d(x,A)>0\) y
\(B(x,\tfrac12 d(x,A))\cap A=\varnothing\).
\\
\hline

\textbf{Distancia entre conjuntos} &
\(d(A,B):=\inf\{d(a,b):a\in A,b\in B\}\).
\; Si \(A,B\) cerrados, disjuntos, y uno compacto \(\Rightarrow d(A,B)>0\).
\\
\hline

\textbf{Contradicciones típicas} &
\textit{Estrategia}: suponer \(x\in \operatorname{int}(A)\) y construir \(r>0\) con \(B(x,r)\not\subset A\).
\; O suponer \(x\notin\overline{A}\) y construir \(r>0\) con \(B(x,r)\cap A=\varnothing\).
\; Usar siempre triangular e inversa.
\\
\hline
\end{tabular}

\section*{Caja de Herramientas Topológicas}

\subsection*{Definiciones clave}
\begin{itemize}
    \item \textbf{Vecindario:} $V$ es vecindario de $x$ si existe un abierto $U$ con $x \in U \subseteq V$.
    \item \textbf{Punto límite:} $x$ es punto límite de $A$ si 
    \[
    \forall U \text{ abierto con } x \in U, \quad (U \setminus \{x\}) \cap A \neq \varnothing.
    \]
    \item \textbf{Clausura:} $\overline{A} = A \cup A'$ (conjunto más todos sus puntos límite).
    \item \textbf{Interior:} 
    \[
    \operatorname{int}(A) = \{x \in A : \exists U \text{ abierto}, \ x \in U \subseteq A\}.
    \]
    \item \textbf{Frontera:} 
    \[
    \partial A = \overline{A} \cap \overline{X \setminus A}.
    \]
\end{itemize}

\subsection*{Propiedades de la clausura}
\begin{itemize}
    \item Extensividad: $A \subseteq \overline{A}$.
    \item Idempotencia: $\overline{\overline{A}} = \overline{A}$.
    \item Monotonía: $A \subseteq B \implies \overline{A} \subseteq \overline{B}$.
    \item Cerrado mínimo: $\overline{A}$ es el menor cerrado que contiene a $A$.
    \item Intersección: $\overline{A \cap B} \subseteq \overline{A} \cap \overline{B}$.
    \item Diferencia: $\overline{A \setminus B} \subseteq \overline{A} \setminus \operatorname{int}(B)$.
\end{itemize}

\subsection*{Propiedades del interior}
\begin{itemize}
    \item Contractividad: $\operatorname{int}(A) \subseteq A$.
    \item Idempotencia: $\operatorname{int}(\operatorname{int}(A)) = \operatorname{int}(A)$.
    \item Monotonía: $A \subseteq B \implies \operatorname{int}(A) \subseteq \operatorname{int}(B)$.
    \item Abierto máximo: $\operatorname{int}(A)$ es el mayor abierto contenido en $A$.
    \item Unión: $\operatorname{int}(A \cup B) = \operatorname{int}(A) \cup \operatorname{int}(B)$.
\end{itemize}

\subsection*{Relaciones importantes}
\begin{itemize}
    \item $\partial A = \overline{A} \cap \overline{X \setminus A}$.
    \item $\overline{A} = \operatorname{int}(A) \cup \partial A$.
    \item $\operatorname{int}(A) = A \setminus \partial A$.
    \item $\partial A = \overline{A} \setminus \operatorname{int}(A)$.
\end{itemize}

\section*{Checklist para demostraciones}

\subsection*{Probar que un conjunto es cerrado}
\begin{enumerate}
    \item Usar la definición: $A$ es cerrado si contiene todos sus puntos límite o si $\overline{A} = A$.
    \item Tomar un punto límite $x$ de $A$.
    \item Mostrar que $x \in A$.
    \item Concluir: $A$ es cerrado.
\end{enumerate}

\subsection*{Probar que un conjunto es abierto}
\begin{enumerate}
    \item Definición: $U$ es abierto si $\forall x \in U, \exists V$ abierto con $x \in V \subseteq U$.
    \item Tomar $x \in U$.
    \item Construir un vecindario abierto dentro de $U$.
    \item Concluir: $U$ es abierto.
\end{enumerate}

\subsection*{Trabajar con clausura}
\begin{enumerate}
    \item Recordar: $\overline{A}$ es el menor cerrado que contiene a $A$.
    \item Para inclusiones: tomar $x \in \overline{A \cap B}$ y usar vecindarios.
    \item Deducir que $x \in \overline{A} \cap \overline{B}$.
\end{enumerate}

\subsection*{Trabajar con interior}
\begin{enumerate}
    \item Tomar $x \in \operatorname{int}(A)$.
    \item Usar que $\exists U$ abierto con $x \in U \subseteq A$.
    \item Encadenar inclusiones según el objetivo.
\end{enumerate}

\subsection*{Trabajar con frontera}
\begin{enumerate}
    \item Tomar $x \in \partial A$.
    \item Usar: todo abierto $U \ni x$ toca $A$ y $X \setminus A$.
    \item Concluir: $x \in \overline{A} \cap \overline{X \setminus A}$.
\end{enumerate}

\end{document}


\end{document}
